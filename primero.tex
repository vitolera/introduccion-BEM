\documentclass[11pt,a4paper]{article}
\usepackage[utf8]{inputenc}
\usepackage[spanish]{babel}
\usepackage{amsmath}
\usepackage{amsfonts}
\usepackage{amssymb}
\author{Victoriano León Ramírez}
\title{Introducción al problema termoelástico aplicado al BEM}
\begin{document}
\maketitle
\section{Planteamiento del problema}
El problema se centra en la termoelasticidad desacoplada, es decir, la incidencia del efecto térmico en el problema elástico y no viceversa. \\
\begin{equation}
 \varepsilon =\varepsilon^{T} +\varepsilon^{E}
 \end{equation}
 \label{hookecalor}
\subsection{El problema térmico}
La ecuación de conducción del calor en un medio tridimensional, isótropo, en estado transitorio y con fuentes de calor interno se define por:
\begin{equation}
\frac{\partial{T}}{\partial{t}}=\frac{\lambda}{\rho c} \nabla^{2} T + \frac{q_{v}}{\rho c}
\end{equation}
donde T representa el incremento de temperatura respecto a un estado $T_{0}$, $\rho$ es la densidad del material, c es el calor específico, $\lambda$ es la conductividad térmica y $q_{v}$ el flujo de calor interno.
Para el caso de no tener fuentes de calor internas eliminaremos el segundo término, y para el problema estacionario $\frac{\partial{T}}{\partial{t}}=0$ por lo que nos quedará que la ecuación de transmisión de calor para el estado estacionario, sin fuentes de calor internas en un medio isótropo tridimensional homogéneo es:
\begin{equation}
\nabla^{2}T=0= \frac{\partial^{2}T}{\partial{x_1^{2}}} +\frac{\partial^{2}T}{\partial{x_2^{2}}} +\frac{\partial^{2}T}{\partial{x_3^{2}}}
\end{equation}
A su vez también podemos escribir la ecuación de expansión del material debido a la dilatación térmica como:
\begin{equation}
\varepsilon^{T}_{ij}=\alpha T \delta_{ij}= \left( \begin{array}{ccc}
\alpha T & 0 & 0 \\
0 & \alpha T & 0\\
0 & 0 & \alpha T \end{array} \right)
\end{equation}
Donde $\alpha$ representa el coeficiente de dilatación térmica.
\subsection{El problema elástico}
En cuanto al problema elástico sus ecuaciones de equilibrio $\nabla \sigma +b =0$ y su ecuación constitutiva $\sigma_{ij}=\lambda \delta_{ij} \varepsilon_{v} + 2 \mu \varepsilon_{ij}$.
\subsection{El problema termoelástico}
Para admitir la formulación del problema termoelástico desacoplado es necesario partir de unas hipótesis: \begin{itemize}
\item Deformaciones pequeñas
\item Incrementos de temperatura pequeños
\item Ecuaciones constitutivas lineales
\end{itemize}
De esta manera podremos asumir la Ley de Hooke Generalizada con efectos térmicos mencionada en (\ref{hookecalor}).
\section{Aplicación al BEM}
A continuación se presentan las ecuaciones integrales de contorno y sus soluciones fundamentales de termoelásticidad obtenidas de \cite{Bal} donde se puede ver su obtención. Se puede apreciar su similitud con las de la elasticidad.
\begin{equation}
\begin{aligned}
c_{ij}(y)u_{j}(y)+\int_{\Gamma}^{CPV} T_{ij}(y,x)u_{j}(x) d \Gamma - \int_{\Gamma} \bar{P_{i}}(y,x)\theta(x) d\Gamma = \\ \int_{\Gamma} U_{ij}(y,x)t_{j}(x) d\Gamma -\int_{\Gamma} \bar{Q_{i}}(y,x)q(x)d\Gamma
\end{aligned}
\end{equation}
donde tenemos como soluciones fundamentales del problema elástico $T_{ij}$ y $U_{ij}$ además de las del problema termoelástico $\bar{Q_{i}}$ y $ \bar{P_{i}}$.
\vspace{0.5cm}
\begin{equation}
\bar{P_{i}}(y,x)=\frac{\alpha(1+\nu)}{8\pi(1-\nu)r}[n_i-\frac{\partial{r}}{\partial{n}}r_{,i}]
\end{equation}
\begin{equation}
\bar{Q_{i}}(y,x)=\frac{-\alpha(1+\nu)}{8\pi\lambda(1-\nu)}[r_{,i}]
\end{equation}
\begin{equation}
T_{ij}(y,x)=\frac{-1}{8\pi(1-\nu)r^2} \left\{\frac{\partial{r}}{\partial{n}}[(1-2\nu)\delta_{ij}+3r_{,i}r_{,j}]-(1-2\nu)(n_j r_{,i}-n_i r_{,j}) \right\}
\end{equation}
\begin{equation}
U_{ij}(y,x)=\frac{-1}{16\pi(1-\nu)\mu r} \{ (3-4\nu) \delta_{ij} + r_{,i} r_{,j}\}
\end{equation}

\begin{thebibliography}{X}
\bibitem{Bal} \textsc{Rafael Balderrama, Manuel Mart{\'\i}nez, Adri{\'a}n P. Casilino },
\textit{Aplicaci{\'o}n De La Integral J De Dominio Al An{\'a}lisis Tridimensional De Grietas En S{\'o}lidos Termoel{\'a}sticos}, Tesis Doctoral,
Caracas, Venezuela, 2004.
\end{thebibliography}
\end{document}